\documentclass{unswmaths}
\usepackage{unswshortcuts}
\usepackage{dsfont}
\begin{document}
\author{Adam J. Gray}
\title{Assignment 1}
\subject{Measure Theory}
\studentno{3329798}

\newcommand{\llra}{\Leftrightarrow}

\unswtitle

\section{}
\subsection{}
Define
\begin{align*}
    \ell_n &= \sum_{k=1}^{N_n} \alpha_k \chi_{C_k} \\
    u_n &= \sum_{k=1}^{N_n} \beta_k \chi_{C_k} 
\end{align*}
where $ \alpha_k := \inf\{ f(x) : x \in C_k \} $ and $ \beta_k := \sup\{ f(x) : x \in C_k \} $ and $ C_k \in \mathcal{P}_n $ where $ \mathcal{P} $ is defined in the question.
We wish to show that $$ \lim_{n\lra\infty} | \ell - f | = 0 = \lim_{n\lra\infty} | f - u_n | $$  $ \lambda $ a.e.

It is obvious that $ \ell \leq f \leq u_n $ and so proving $$ \lim_{n\lra\infty} |u_n - \ell_n| = 0 $$ is sufficient.
Define $$ \phi_n := u_n - \ell_n = \sum_{k=1}^{N_n} (\beta_k - \alpha_k) \chi_{C_k}. $$
Firstly we must show $ \lim_{n\lra\infty} \phi_n $ exists $ \lambda $ a.e. This follows from the fact that $ \ell_n $ ( $u_n$ ) is a non-decreasing (non-increasing) sequence which is bounded above (below) by $ f $ which is also bounded. 

We now wish to establish that $ \lim_{n\lra\infty} \phi_n = 0 $. Note that 
$$
    \phi_n \leq \sup\{ f(x)-f(y) : x,y \in S \} \leq K\chi_{S}
$$
for some $ K $ because $ f $ is bounded. Now because $ S $ is bounded in $ \mathbb{R}^d $ we have that $ K\chi_S \in \mathcal{L}^1(S) $. The dominated convergence theorem therefore allows us to write  
\begin{align*}
    \int \lim_{n\lra\infty} \underbrace{\sum_{k=1}^{N_n}(\beta_k - \alpha_k) \chi_{C_k}}_{=\phi_n} d\lambda &=
        \lim_{n \lra\infty} \int \sum_{k=1}^{N_n} (\beta_k - \alpha_k) \chi_{C_k} d\lambda \\
    &=  \lim_{n \lra\infty} \sum_{k=1}^{N_n} (\beta_k - \alpha_k) \lambda(C_k) \text{ because } \phi_n \text{ is a simple function }\\
    &= 0 \text{ because } f \text{ is Riemann integrable }.
\end{align*}
This means that $ \lim_{n\lra\infty} \phi_n = 0 $, $ \lambda $ a.e. and hence $ \lim_{n\lra\infty} \ell_n = f = \lim_{n\lra\infty} \ell_n $, $ \lambda $ a.e.

We now show that the Riemann integral and the Lebesgue integral coincide. 
We have that 
$$
    |\ell_n| \leq M \chi_{S} 
$$
for some $ M $ because $ f $ is bounded. By the same argument as above $ M \chi_{S} \in \mathcal{L}^1(S) $. $ \lim_{n\lra\infty} \ell_n $ exists and equals $ f $ (this was established above), so by the dominated convergence theorem,
\begin{align*}
    \underbrace{\lim_{n\lra\infty} \int \ell_n d\lambda}_{\int_{S} f(x) dx} &= \int \underbrace{\lim_{n\lra\infty} \ell_n}_{\circledast} d\lambda \\
    &= \underbrace{\int f d\lambda}_{\text{Lebesgue integral}} \\
\end{align*}

\subsection{}
Let $ N = \{ x : \lim_{n\lra\infty}\ell_n(x) \neq \lim_{n\lra\infty}u_n(x) \} $ and
let $ X = S / (\partial S \cup N ) $. 
We wish to show that for all $ x \in X $ and $ \varepsilon > 0 $ there exists a $ \delta > 0 $ such that
$$ \sup\{ f(x_1) - f(x_2) : ||x-x_1|| < \delta, ||x-x_2|| < \delta \} < \varepsilon. $$
Fix $ x $ and $ \varepsilon $. As $ \lim_{n\lra\infty} (u_n - \ell_n) = 0 $ there exists an $ L $ such that $ n > L $ implies $ u_n - \ell_n < \varepsilon $
which is to say that 
\begin{align*}
    \sum_{k=1}^{N_n} (\beta_k - \alpha_k) \chi_{C_k} < \varepsilon
\end{align*}
which implies that
\begin{align*}
    (\beta_k - \alpha_k) < \varepsilon
\end{align*}
where $ x \in C_k $. Therefore chosing $ \delta = \inf\{ ||y-x|| : y \not\in C_k \} $ guarentees
$ \{ y : ||y - x || < \delta \} \subseteq C_k $ and hence $$ \sup\{ f(x_1) - f(x_2) : ||x-x_1|| < \delta, ||x-x_2|| < \delta \} < \varepsilon. $$
We now need to prove that $ \delta > 0 $. To do this we show that there exits a partitioning of $ S $ such that $$ \underbrace{\inf\{ || x_1 - x || : x_1 \not\in C_k \} > 0}_{\circledast}$$ for all $ x \in X $ so long as $ x \in C_k / (\partial C_k ) $. We then show that there always exists a partitioning of $ X $ such that $ x \not\in \partial C_k $ and the result will follow. 

A partitioning of $ S $ such that $ \circledast $ holds for all $ n $ is given by
\begin{align}
    C_k = \Big( \frac{k(b_1 - a_1)}{q^n}, \frac{(k+1)(b_1 - a_1)}{q^n}\Big] \times \cdots \times \Big( \frac{k(b_d-a_d)}{q^n}, \frac{(k+1)(b_d-a_d)}{q^n} \Big]
\end{align}
where $ q = 2 $. Now suppose that $ x \in \partial C_k $ then the $ \circledast $ would still hold when $ q = 3 $ but $ x \not\in \partial C_k $. 

\section{}
\subsection{}
Note that for $ E \in \mathcal{B} $
\begin{align*}
    \left(f^{-1}(E)\right)^c &= \left(\{ x \in X : f(x) \in E \}\right)^c \\
    &= \{ x \in X : f(x) \not\in E \} \\
    &= \{ x \in X : f(x) \in E^c \} \\
    &= f^{-1}(E^c).
\end{align*}
So if $ A \in \mathcal{A} $ then $ A^c \in \mathcal{A} $.

Also note that for $ E_1, E_2, \ldots $ 
we have that
\begin{align*}
    \bigcup_{i} f^{-1}(E_i) &= \bigcup_{i} \{ x \in X : f(x) \in E_i \} \\
        &= \left\{ x \in X : f(x) \in \bigcup_i E_i \right\} \\
        &= f^{-1}\left(\bigcup_i E_i \right).
\end{align*}
which means that if $ F_1, F_2, \ldots $ is some countable collection of sets in $ \mathcal{A} $ then their union is also in $ \mathcal{A} $.

Finnaly note that $ f^{-1}(Y) = X $. Therefore $ \mathcal{A} := \{ f^{-1}(B) : B \in \mathcal{B} \} $ is a $ \sigma$-algebra. 

Clearly if we remove any set from $ \mathcal{A} $ then $ f $ would not be $ \mathcal{A} $ measurable because there would exist a set $ B \in \mathcal{B} $ such that $ f^{-1}(B) \not\in \mathcal{A} $. 

\subsection{}
Define
\begin{align*}
    g : Y \lra Z \\
    g := a_n\mathds{1}_{B_n}
\end{align*}

Now suppose $ E \in \mathcal{C} $ and note that
\begin{align*}
    h^{-1}E &= f \circ h^{-1} E \\
        &= f \bigcup_{n \in I} \underbrace{h^{-1} a_n}_{\circledast}
\end{align*}
where $ I = \{ n : a_n \in E \}  $.

$ \circledast $ is because if $ E \in \mathcal{C} $ then if $ U := \{ a_n  : n \in I, I \subseteq \mathbb{Z} \} \subseteq E $ then 
$$ h^{-1} E = h^{-1} \left(U \cup \left(E \setminus U \right)\right) = h^{-1} \bigcup_{n\in I} a_n = \bigcup_{n \in I} h^{-1} a_n. $$

Now because $ h $ is measurable then 
$$
    D := \bigcup_{n \in I} h^{-1}a_n
$$
is the \emph{countable} union of measurable sets, hence measurable.
Then we just have to argue that
$$
    fD \in Y.
$$
This is clear because $ D = f^{-1} E $ for some $ E \in \mathcal{B} $ and therefore $ fD = ff^{-1}E = E \in \mathcal{B} $.


\section{}
\subsection{}
$ i \Longrightarrow ii $:

Obvious. (By definition)

$ ii \Longrightarrow iii $:

\begin{align*}
    & \liminf_{n}\chi_{A_n}(x) = 1 \\
    &\Longrightarrow  \lim_{N\lra\infty} \inf_{n \geq N} \chi_{A_n}(x) = 1\\
    &\Longrightarrow  \chi_{A_n}(x) = 0 \text{ for finitely many } n \\ 
    &\Longrightarrow x \not\in A_n \text{ for finitely many } n 
\end{align*}
 
$ iii \Longrightarrow i $:

\begin{align*}
    & x \not\in A_n \text{ for finitely many } n \\
    & \Longrightarrow \exists N \text{ such that } n \geq N \Longrightarrow x \in A_{n} \\
    & \Longrightarrow n\geq N \Longrightarrow x \in \bigcap_{k\geq n} A_n \\
    & \Longrightarrow x \in \bigcup_{n} \bigcap_{k\geq n} A_{k} \\
\end{align*}

\subsection{}

\begin{align*}
    x \in \limsup_{n} A_n \\
    \limsup_{n} \chi_{A_n}(x) = 1\\
    x \in A_n \text{ for infinitely many } n 
\end{align*}
Clearly the third condition here is less restrictive than the third contion in 3.1

\section{}
\subsection{}
Clearly
\begin{align*}
    \emptyset \in \mathcal{A} \Longrightarrow \emptyset \in \mathcal{A}_{c} \\
\end{align*}
also if we let $ A \in \mathcal{A}_{c} $
\begin{align*}
    A^{c} &:= A^{c} \cap C
\end{align*}
and $ A^{c} \cap C \in \mathcal{A}_{c} $ because $ A^{c} \in \mathcal{A} $.

Now let $ \{ A_n \}_n \subseteq \mathcal{A}_{c} $ where $ A_n = A'_n \cap C $ with $ A'_n \in \mathcal{A} $ then 
\begin{align*}
    \bigcup_{n} A_n &= \bigcup_{n} (A'_n \cap C) \\
        &= \bigcup_{n} (A'_n \cap C) \\
        &= C \cap \underbrace{\bigcup_{n} (A'_n)}_{\in \mathcal{A}}
\end{align*}
and thus 
$$
    \bigcup_{n} A_n \in \mathcal{A}_{c}.
$$
\end{document}
