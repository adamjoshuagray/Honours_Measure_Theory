\documentclass{unswmaths}
\usepackage{unswshortcuts}
\begin{document}
\author{Adam J. Gray}
\title{Assignment 1}
\subject{Measure Theory}
\studentno{3329798}

\newcommand{\llra}{\Leftrightarrow}

\unswtitle

\section{}
\subsection{}
Define
\begin{align*}
    \ell_n &= \sum_{k=1}^{N_n} \alpha_k \chi_{C_k} \\
    u_n &= \sum_{k=1}^{N_n} \beta_k \chi_{C_k} 
\end{align*}
where $ \alpha_k := \inf\{ f(x) : x \in C_k \} $ and $ \beta_k := \sup\{ f(x) : x \in C_k \} $ and $ C_k \in \mathcal{P}_n $ where $ \mathcal{P} $ is defined in the question.
We wish to show that $$ \lim_{n\lra\infty} | \ell - f | = 0 = \lim_{n\lra\infty} | f - u_n | $$  $ \lambda $ a.e.

It is obvious that $ \ell \leq f \leq u_n $ and so proving $$ \lim_{n\lra\infty} |u_n - \ell_n| = 0 $$ is sufficient.
Define $$ \phi_n := u_n - \ell_n = \sum_{k=1}^{N_n} (\beta_k - \alpha_k) \chi_{C_k}. $$
Firstly we must show $ \lim_{n\lra\infty} \phi_n $ exists $ \lambda $ a.e. This follows from the fact that $ \ell_n $ ( $u_n$ ) is a non-decreasing (non-increasing) sequence which is bounded above (below) by $ f $ which is also bounded. 

We now wish to establish that $ \lim_{n\lra\infty} \phi_n = 0 $. Note that 
$$
    \phi_n \leq \sup\{ f(x)-f(y) : x,y \in S \} \leq K\chi_{S}
$$
for some $ K $ because $ f $ is bounded. Now because $ S $ is bounded in $ \mathbb{R}^d $ we have that $ K\chi_S \in \mathcal{L}^1(S) $. The dominated convergence theorem therefore allows us to write  
\begin{align*}
    \int \lim_{n\lra\infty} \underbrace{\sum_{k=1}^{N_n}(\beta_k - \alpha_k) \chi_{C_k}}_{=\phi_n} d\lambda &=
        \lim_{n \lra\infty} \int \sum_{k=1}^{N_n} (\beta_k - \alpha_k) \chi_{C_k} d\lambda \\
    &=  \lim_{n \lra\infty} \sum_{k=1}^{N_n} (\beta_k - \alpha_k) \lambda(C_k) \text{ because } \phi_n \text{ is a simple function }\\
    &= 0 \text{ because } f \text{ is Riemann integrable }.
\end{align*}
This means that $ \lim_{n\lra\infty} \phi_n = 0 $ and hence $ \lim_{n\lra\infty} \ell_n = f = \lim_{n\lra\infty} \ell_n $.

We now show that the Riemann integral and the Lebesgue integral coincide. 
We have that 
$$
    |\ell_n| \leq M \chi_{S} 
$$
for some $ M $ because $ f $ is bounded. By the same argument as above $ M \chi_{S} \in \mathcal{L}^1(S) $. $ \lim_{n\lra\infty} \ell_n $ exists and equals $ f $ (this was established above), so by the dominated convergence theorem,
\begin{align*}
    \underbrace{\lim_{n\lra\infty} \int \ell_n d\lambda}_{\int_{S} f(x) dx} &= \int \underbrace{\lim_{n\lra\infty} \ell_n}_{\circledast} d\lambda \\
    &= \underbrace{\int f d\lambda}_{\text{Lebesgue integral}} \\
\end{align*}

\subsection{}


end{document}
