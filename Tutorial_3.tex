\documentclass{unswmaths}
\usepackage{unswshortcuts}
\begin{document}
\author{Adam J. Gray}
\title{Tutorial 3}
\subject{Measure Theory}
\studentno{3329798}

\newcommand{\llra}{\Leftrightarrow}

\unswtitle

\section{}
\subsection{}
\subsubsection{$ f + g $}

\begin{align*}
    f(x) + g(x) < t \llra f(x) < r \text{ and } g(x) < t + r
\end{align*}
for some $ r \in \mathbb{Q} $.

Thus
\begin{align*}
    \{ x : f(x) + g(x) < t \} = \bigcup_{r \in \mathbb{Q}} \left[ f^{-1}([-\infty, r)) \cap g^{-1}([-\infty, t-r)) \right].
\end{align*}
This is the countable union of measurable sets and is thus measurable itself. So $ f + g $ is measurable.

\subsubsection{$ \alpha f $}

\begin{align*}
    \{ x : \alpha f(x) < t \} = \bigcup_{q \in \mathbb{Q}, q < \alpha} \left[ f^{-1}([-\infty, \frac{t}{\alpha})) \right].
\end{align*}
This is the countable union of measurable sets and is thus measurable itself. So $ \alpha f $ is measurable.

\subsubsection{$ fg $}

\begin{align*}
    fg = \frac{(f + g)^{2} - f^2 - g^2}{2}
\end{align*}
and
\begin{align*}
    \{ x : f^{2}(x) < t \} = \{ x : f(x) < \sqrt{t} \} = \bigcup_{q \in \mathbb{Q}, q < \sqrt{t}} f^{-1}([-\infty, q))
\end{align*}
This is the countable union of measurable sets so it is measurable and so $ f^2 $ is measurable. The measurability of $ fg $ follows from this and the last two results.

\subsubsection{$ f/g $}

\begin{align*}
    \{ x : \frac{1}{g(x)} < t \} = 
    \begin{cases}
	\{ x : \frac{1}{t} < g(x) < 0 \} & t < 0 \\
	\{ x : -\infty < g(x) < 0 \} & t = 0\\
	\{ x : -\infty < g(x) < 0 \} \cup \{ x : \frac{1}{t} < g(x) < 0 \} & b > 0
    \end{cases}
\end{align*}

Which is clearly measurable so $ \frac{1}{g} $ is measurable and thus $ \frac{f}{g} $ is measurable.

\subsubsection{$ f \wedge g $}

$$ f \wedge g = \frac{f + g + |f-g| }{2} $$
and
$$
    \{ x : |f(x)| < t \} = \bigcup_{q \in \mathbb{Q}, q < t} f^{-1}([-\infty, q)) \cap f^{-1}((-q, \infty]) 
$$
which is the countable union of measurable sets so $ |f| $ is measurable. It follows that $ f \wedge g $ is measurable by previous results.

\subsubsection{$ f \vee g $}

$$ f \vee g = \frac{f + g - |f-g| }{2}. $$ The result then follows from previous results.

\subsection{}

\subsubsection{$\{ x : f(x) < g(x) \}$}

$ \exists r : f(x) < r < g(x) $ for each $ x $.

$$
    \{ x : f(x) < g(x) \} = \bigcup f^{-1}([-\infty, r)) \cap g^{-1}((r, \infty])
$$
which is the countable union of measurable sets so $ \{ x : f(x) < g(x) \} $ is measurable.

\subsubsection{$\{ x : f(x) \leq g(x) \}$}

$$ \{ x : f(x) \leq g(x) \} = \{ x : g(x) < f(x) \}^{c} $$ and the result follows from above because the complement of a measurable set is measurable.

\subsubsection{$\{ x : f(x) = g(x) \}$}

$\{ x : f(x) = g(x) \} = \{ x : f(x) < g(x) \} \cap \{ x : f(x) \leq g(x) \} $ and the result follows from finite intersections.

\subsection{}
\subsubsection{ $ \sup_n f_n $ }

$$
    \{ \sup_n f_n < t \} = \bigcap_{n \in \Ntrl} \{x : f_n(x) < t \}
$$
which is the countable intersection of measurable sets and therefore $ \sup_n f_n $ is measurable.

\subsubsection{ $ \inf_n f_n $ }

Likewise 
$$
    \{ \inf_n f_n < t \} = \bigcup_{n \in \Ntrl} \{ x : f_n(x) < t \} 
$$
and so by the same argument with unions the result follows. 

\subsubsection{$ \limsup_n f_n $}
$$
    \limsup_n f_n = \inf_k \sup_{n\geq k} f_n
$$
and the result follows immediatly by putting $ \sup_{n \geq k} f_n $ in the previous result and noting that $ \sup_{n \geq k} f_n$ 
is measurable by the result for $ \sup$ .

\subsubsection{$ \liminf_n f_n $}
$$
    \liminf_n f_n = \sup_k \inf_{n\geq k} f_n
$$
and using an essentially identical argument as in the previous result.
\section{}

\subsection{}
\begin{align*}
    F(t) &= \mu\left( \{ x : f(x) > t \} \right) \\
    F(t+\varepsilon) &= \mu\left( \{ x : f(x) > t + \varepsilon \} \right)
\end{align*}
and as
\begin{align*}
    \{ x : f(x) > t + \varepsilon \} \subseteq \{ x : f(x) > t \}
\end{align*}
then
\begin{align*}
  \mu\left( \{ x : f(x) > t + \varepsilon \} \right) \leq \mu\left( \{ x : f(x) > t \} \right)
\end{align*}
and so $ F(t+\varepsilon) \leq F(t) $.

Also as $ f $ is a bounded function there must exist an $ N $ such that $ f(x) \leq N $ $ \forall $ $ x $, which means
$ \{ x : f(x) > N  \} = \emptyset $ and so $ F(t) = 0 $ $ \forall $ $ t > N $.

Note that $ \lim_{t \lra 0} F(t) $ is finite if $ f $ is of finite support.

\begin{align*}
    \lim_{n \lra \infty} \int_0^n F(t)dt &= \int_0^N F(t) dt + \underbrace{\lim_{n \lra \infty} \int_N^n F(t) dt}_{=0} 
	&\leq N F(0).
\end{align*}

Now as $ N $ and $ F(0) $ are finite then $ \lim_{n \lra \infty} \int_0^n F(t)dt < \infty $. 
Now as $ F(t) $ is non-negative then $ G(n) = \int_0^n F(t) dt $ is an increasing function, bounded by $ NF(0) $ and thus,
$ \lim_{n \lra \infty} G(n) $ exists.

\subsection{}
$$
    L_n = \sum_{k = 1}^{N2^n} \frac{1}{2^n} F(\frac{k}{2^n})
$$
which is clearly the Lebesgue integral of
$$
    S_n(x) = \sum_{k=1}^{N2^n} \frac{1}{2^n} \chi_{\{ t : f(t) > \frac{k}{2^n} \}}(x).
$$
\subsection{}
It is also clear that
$$
    S_n(x) = \frac{1}{2^n} \lfloor 2^nf(x) \rfloor
$$
and so $ \lim_{n \lra \infty} S_n(x) = f(x) $. Also by noting that 
$$
    \frac{1}{2} \lfloor 2^{n+1} f(x) \rfloor \geq \lfloor 2^n f(x) \rfloor 
$$
we have that $ S_{n+1}(x) \geq S_n(x) $.
Thus 
$$
    \lim_{n \lra \infty} \int S_n d\mu = \int f d\mu.
$$
by the monotone convergence theorem.
Now as $L_n$ was also the lower Riemann integral of $F$ and as $ n \lra \infty $, $ L_n \lra \int_0^N F(x) dx = \int_0^\infty F(x) dx $ then 
$$
    \int f d\mu = \int_0^\infty F(x) dx.
$$

\section{}

Let 
$$
    f_n(x) = \chi_{A_n}(x) f(x) + n \chi_{A_n^c}(x)
$$
then
$$
    \int f d \mu = \int_{A_n} f d\mu + \int_{A_n^c} f d\mu > \int f_n d\mu + n \mu(A_n^c).
$$

Then as clearly $ f_n \leq f $ and $ \lim_{n \lra \infty} f_{n}(x) = f(x) $ and $ f \in \mathcal{L}^1(\mu) $,by the dominated convergence theorem 
$$
    \lim_{n \lra \infty} \int f_n d \mu = \int f d \mu
$$
which implies that $ \lim_{n\lra \infty} n \mu( A_n^c) = 0 $

\end{document}
