\documentclass{unswmaths}
\usepackage{unswshortcuts}
\usepackage{dsfont}
\usepackage{fullpage}
\begin{document}
\author{Adam J. Gray}
\title{Assignment 2}
\subject{Measure Theory}
\studentno{3329798}

\newcommand{\llra}{\Leftrightarrow}

\unswtitle

\section{}
\section{}
\subsection{}
Suppose $ u $ and $ \nu $ are $\sigma$-finite \emph{positive} measures on $ (\Omega, \mathcal{F}) $. Then suppose that $ \mu << \nu $ and $ \nu << \mu $.
Then for $ A \in \mathcal{F} $ we have that $ \mu(a) = 0 \Rightarrow \nu(A) = 0 $ and $ \nu(A) = 0 \Rightarrow \mu(A) $. That is to say that $ \nu(A) =0 \Leftrightarrow \mu(A) = 0 $. That is to say that $ \nu $ and $ \mu $ have the same null sets. This argument is symmetric so it is clear that the reverse implication also holds.

We now wish to show that there is an $ \mathcal{F}$-measurable function $ g $ that satisfies $ 0 < g(\omega) < +\infty $ at each $ \omega \in \Omega $ and is such that $ \nu(A) = \int_Agd\mu $ for all $ A \in \mathcal{F} $.

\section{}
\subsection{}
Firstly note that the definition of the characteristic function is
\begin{align}
	\hat{\mu}_X(u) = \mathbb{E}[\exp(i\langle X, u \rangle)]
\end{align}
and so for the random vector $ cX $ with $ c \in \Rl $ we have that
\begin{align}
	\hat{\mu}_{cX}(u) &= \mathbb{E}[ \exp(i \langle cX, u \rangle) ] \\
		 &= \mathbb{E}[ \exp(i \langle X, cu \rangle) ] \\
		 &= \hat{\mu}_{X}(cu)
\end{align}
\subsection{}
\subsection{}
Let $ d = 1 $ and let $ \mu $ have the  Lebesgue density,
\begin{align}
	f(x) = \frac{C}{(1+x^2) \log(e + x^2)}, \ \ \ \ x \in \Rl.
\end{align}
We wish to show that $ E[X] $ is not defined but $ \hat{\mu}(u) $ is differentiable at $ 0 $. Firstly we show that $ E[X] $ is not defined. 

\begin{align}
	E[X] &= \int_{-\infty}^\infty \frac{Cx}{(1+x^2) \log(e + x^2)} dx \\
		&= \int_{-\infty}^\infty \frac{C}{(x^{-1} + x)\log(e + x^2)}
\end{align}
note that
\begin{align}
	\frac{C}{(x^{-1} + x)\log(e + x^2)} \sim \frac{C}{2x\log(x)}
\end{align}
and 
\begin{align}
	\int_{a}^\infty \frac{C}{2x\log(x)}dx, \ \text{and} \ \int_{-\infty}^b \frac{C}{2x\log(x)}dx
\end{align}
do not converge so $ \mathbb{E}[X] $ does not exist.

We now just have to show that $ \hat{mu}(u) $ is differentiable at $ u = 0 $. 

\begin{align}
	\hat{\mu}(u) = \int_{-\infty}^\infty \frac{e^{iux}C}{(1+x^2)\log(e+x^2)} dx
\end{align}
and
\begin{align}
	\frac{d}{du}\hat{\mu}(u) \Big|_{u = 0} = \int_{-\infty}^\infty \frac{ixe^{iux}C}{(1+x^2)\log(e+x^2)} dx
\end{align}

\section{}
Let $\mu$ be the binomial distribution with $ n $ trials and probability of success $ p $, that is $ \mu = \operatorname{Bin}(n,p) $, and let $ \nu $ be the Poisson distribution with mean $ \lambda > 0 $. 
\subsection{}
We wish to verify that $ \hat{\mu}(u) = (1 - p + pe^{iu})^n $. 
Because the binomial distribution is just the convolution of identical independent Bernoulli distributions then we just have to verify that $ (1 - p + pe^{iu} $ is the characteristic function for $Bernoulli(p)$.

If $ \nu $ is the Bernoulli measure and $ X $ has law $ \nu $ then 
\begin{align}
	\hat{\nu}(u) &= \mathbb{E}[ \exp(iuX)] \\
		&= \sum_{k \in \{ 0, 1\}} e^{iuk} \nu_{X}(k) \\
		&=  pe^{iu} + (1 - p).
\end{align}
Then by repeated application of the convolution theorem we get that
$ \hat{\mu}(u) = (1 - p + pe^{iu})^n $.

\subsection{}
We wish to verify that $ \hat{\nu}(u) = \exp(\lambda(e^{iu} - 1)) $. 
The probability mass function of the Poisson distribution is 
\begin{align}
	\frac{\lambda^k}{k!} e^{-\lambda} 
\end{align}
and thus
\begin{align}
	\mathbb{E}[ \exp(iuX) ] &= \sum_{k=0}^\infty \frac{\lambda^k}{k!} e^{-\lambda} e^{iuk} \\
		&= e^{-\lambda}\sum_{k=0}^\infty \frac{\lambda^k}{k!} (e^{iu})^k \\
		&= e^{-\lambda}\sum_{k=0}^\infty \frac{(\lambda e^{iu})^k}{k!} \\
		&= e^{-\lambda} e^{\lambda e^{iu}} \\
		&= e^{\lambda( e^{iu} - 1)}
\end{align}

\subsection{}

We wish to show that if $ p_n $ is a sequence in $ [0,1] $ such that $ p_n \downarrow 0 $ and $ np_n \lra \lambda $ then $ \mu_n \lra \nu $ in the weak sense where $ \mu_n = \operatorname{Bin}(n, p_n) $.

Let $ f \in C_b $ then 
\begin{align}
	E[f(X_n)] &= \sum_{k=0}^n f(k) \mu_n(k) \\
		&= \sum_{k=0}^n f(k) \mu_n(k) \\
		&= \sum_{k=0}^n f(k) {n\choose k} p_n^k(1-p_n)^{n-k}
\end{align}
then
\begin{align}
	\lim_{n\lra\infty} E[f(X_n)] &= \lim_{n \lra \infty}\sum_{k=0}^n f(k) {n\choose k} p_n^k(1-p_n)^{n-k}
\end{align}
and
\begin{align}
	\mathbb{E}[f(Y)] &= f(x) \nu(x) \\
		&= \sum_{k=0}^\infty f(k) \frac{\lambda^k}{k!} e^{-\lambda} \\
\end{align}

\section{}
\subsection{}
We wish to show that $ \mathbb{P}(B_n) = 1/2 $ for every $ n \geq 1 $.
Note that 
\begin{align}
	B_n = \bigcup_{k=0}^{2^{n-1} - 1} \Big[ \frac{2k}{2^n}, \frac{2k+1}{2^n} \Big)
\end{align}
and thus
\begin{align}
	\mathbb{P}(B_n) &= \mathbb{P}\left( \bigcup_{k=0}^{2^{n-1} - 1} \Big[ \frac{2k}{2^n}, \frac{2k+1}{2^n} \Big) \right) \\
		&= \sum_{k=0}^{2^{n-1}-1} \mathbb{P}\left( \Big[ \frac{2k}{2^n}, \frac{2k+1}{2^n} \Big) \right) \\
		&= \sum_{k=0}^{2^{n-1}-1} \frac{1}{2^n} \\
		&= 2^{n-1} \frac{1}{2^n} \\
		&= \frac{1}{2}.
\end{align}
\subsection{}
We now wish to show that the sequence of events $ B_n $ form an infinite sequence of independent events. 

Take a finite subset $ J \subset \Ntrl $ with $ |J| = m $ and $ \max J = r $ then
\begin{align}
	\mathbb{P}\left( \bigcap_{n \in J} B_n \right) &= \mathbb{P}\left( \bigcap_{n \in J} \bigcup_{k=0}^{2^{n-1} - 1} \Big[ \frac{2k}{2^n}, \frac{2k+1}{2^n} \Big) \right) \\
	&= \mathbb{P} \left( \bigcup_{k=0}^{2^{r-m} - 1} \Big[ \frac{2k}{2^r}, \frac{2k+1}{2^r} \Big)\right) \\
	&=  \sum_{k=0}^{2^{r-m} - 1} \mathbb{P} \left( \Big[ \frac{2k}{2^n}, \frac{2k+1}{2^n} \Big) \right) \\
	&= 2^{r-m} \frac{1}{2^r} \\
	&= \frac{1}{2^m} \\
	&= \prod_{n \in J} \mathbb{P}\left( B_n \right)
\end{align}
and so the sequence of events $ B_n $ form an infinite sequence of independent events.
\subsection{}




\end{document}